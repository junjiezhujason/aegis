\documentclass[a4paper,10pt,BCOR10mm,oneside,headsepline]{scrartcl}
% \usepackage[paperheight=9in, paperwidth=8in, margin=0pt]{geometry}

% general
\usepackage{amsmath,amssymb,graphicx,tikz,psfrag}
% for algorithms
\usepackage[linesnumbered,ruled,vlined]{algorithm2e}
\usepackage{algorithmic}
% for captions
\renewcommand{\thealgocf}{} % removes caption numbering
% for sizing
\usepackage[
    % paperheight=11in,
    lmargin=2cm,
    rmargin=2cm,
    tmargin=2.00cm,
    bmargin=1.50cm]{geometry}

\usepackage{scalerel}
\newcommand{\scalesize}{1.3}
% more configurations
\thispagestyle{empty} % to turn off the page number
\nopagecolor % removes the pdf background color

\usepackage[ngerman]{babel}
\usepackage[utf8]{inputenc}
\usepackage{wasysym}% provides \ocircle and \Box
\usepackage{enumitem}% easy control of topsep and leftmargin for lists
\usepackage{color}% used for background color
\usepackage{forloop}% used for \Qrating and \Qlines
\usepackage{ifthen}% used for \Qitem and \QItem
\usepackage{typearea}
% \areaset{17cm}{26cm}
% \setlength{\topmargin}{-1cm}
\usepackage{scrpage2}
\pagestyle{scrheadings}
\ohead{\pagemark}
\chead{}
\cfoot{}

%%%%%%%%%%%%%%%%%%%%%%%%%%%%%%%%%%%%%%%%%%%%%%%%%%%%%%%%%%%%
%% Beginning of questionnaire command definitions %%
%%%%%%%%%%%%%%%%%%%%%%%%%%%%%%%%%%%%%%%%%%%%%%%%%%%%%%%%%%%%
%%
%% 2010, 2012 by Sven Hartenstein
%% mail@svenhartenstein.de
%% http://www.svenhartenstein.de
%%
%% Please be warned that this is NOT a full-featured framework for
%% creating (all sorts of) questionnaires. Rather, it is a small
%% collection of LaTeX commands that I found useful when creating a
%% questionnaire. Feel free to copy and adjust any parts you like.
%% Most probably, you will want to change the commands, so that they
%% fit your taste.
%%
%% Also note that I am not a LaTeX expert! Things can very likely be
%% done much more elegant than I was able to. If you have suggestions
%% about what can be improved please send me an email. I intend to
%% add good tipps to my website and to name contributers of course.
%%
%% 10/2012: Thanks to karathan for the suggestion to put \noindent
%% before \rule!

%% \Qq = Questionaire question. Oh, this is just too simple. It helps
%% making it easy to globally change the appearance of questions.
\newcommand{\Qq}[1]{\textbf{#1}}

%% \QO = Circle or box to be ticked. Used both by direct call and by
%% \Qrating and \Qlist.
\newcommand{\QO}{$\Box$}% or: $\ocircle$

%% \Qrating = Automatically create a rating scale with NUM steps, like
%% this: 0--0--0--0--0.
\newcounter{qr}
\newcommand{\Qrating}[1]{\QO\forloop{qr}{1}{\value{qr} < #1}{------\QO}}

%% \Qline = Again, this is very simple. It helps setting the line
%% thickness globally. Used both by direct call and by \Qlines.
\newcommand{\Qline}[1]{\noindent\rule{#1}{0.6pt}}

%% \Qlines = Insert NUM lines with width=\linewith. You can change the
%% \vskip value to adjust the spacing.
\newcounter{ql}
\newcommand{\Qlines}[1]{\forloop{ql}{0}{\value{ql}<#1}{\vskip0em\Qline{\linewidth}}}

%% \Qlist = This is an environment very similar to itemize but with
%% \QO in front of each list item. Useful for classical multiple
%% choice. Change leftmargin and topsep accourding to your taste.
\newenvironment{Qlist}{%
\renewcommand{\labelitemi}{\QO}
\begin{itemize}
}{%
\end{itemize}
}

%% \Qtab = A "tabulator simulation". The first argument is the
%% distance from the left margin. The second argument is content which
%% is indented within the current row.
\newlength{\qt}
\newcommand{\Qtab}[2]{
\setlength{\qt}{\linewidth}
\addtolength{\qt}{-#1}
\hfill\parbox[t]{\qt}{\raggedright #2}
}

%% \Qitem = Item with automatic numbering. The first optional argument
%% can be used to create sub-items like 2a, 2b, 2c, ... The item
%% number is increased if the first argument is omitted or equals 'a'.
%% You will have to adjust this if you prefer a different numbering
%% scheme. Adjust topsep and leftmargin as needed.
\newcounter{itemnummer}
\newcommand{\Qitem}[2][]{% #1 optional, #2 notwendig
\ifthenelse{\equal{#1}{}}{\stepcounter{itemnummer}}{}
\ifthenelse{\equal{#1}{a}}{\stepcounter{itemnummer}}{}
\begin{enumerate}
\item[\textbf{\arabic{itemnummer}#1.}] #2
\end{enumerate}
}

%% \QItem = Like \Qitem but with alternating background color. This
%% might be error prone as I hard-coded some lengths (-5.25pt and
%% -3pt)! I do not yet understand why I need them.
\definecolor{bgodd}{rgb}{0.8,0.8,0.8}
\definecolor{bgeven}{rgb}{0.9,0.9,0.9}
\newcounter{itemoddeven}
\newlength{\gb}
\newcommand{\QItem}[2][]{% #1 optional, #2 notwendig
\setlength{\gb}{\linewidth}
\addtolength{\gb}{-5.25pt}
\ifthenelse{\equal{\value{itemoddeven}}{0}}{%
\noindent\colorbox{bgeven}{\hskip-3pt\begin{minipage}{\gb}\Qitem[#1]{#2}\end{minipage}}%
\stepcounter{itemoddeven}%
}{%
\noindent\colorbox{bgodd}{\hskip-3pt\begin{minipage}{\gb}\Qitem[#1]{#2}\end{minipage}}%
\setcounter{itemoddeven}{0}%
}
}

%%%%%%%%%%%%%%%%%%%%%%%%%%%%%%%%%%%%%%%%%%%%%%%%%%%%%%%%%%%%
%% End of questionnaire command definitions %%
%%%%%%%%%%%%%%%%%%%%%%%%%%%%%%%%%%%%%%%%%%%%%%%%%%%%%%%%%%%%

\begin{document}

\begin{center}
\textbf{\huge Feedback Questionnaire for AEGIS}
\end{center}
\vskip2em

\minisec{Please tell us a little bit about yourself below.}
\vskip.5em

\Qitem[a]{ \Qq{What is your research area?} \Qline{10.6cm} .}

\Qitem[b]{ \Qq{Do you study either genes or gene products?} \hskip0.4cm  \QO{} No. \hskip0.4cm \QO{}
Yes, I study them for \Qline{2cm} .}


\Qitem[c]{ \Qq{Have you used any tools for GO (Gene Ontology) analysis?}
\begin{Qlist}
\item No, I do not know what GO is.
\item No, I know what GO is, but never used any GO analysis tools.
\item Yes, I have used the following platforms: \Qline{8.5cm} .
\end{Qlist}
}

\minisec{Please evaluate the following user manuals: \texttt{http://aegis.stanford.edu/manual/}.}
\vskip.5em

\Qitem[a]{ \Qq{Visualization panel:} \Qtab{5cm}{difficult to follow ~~ \Qrating{5} ~~ easy to follow}}

\Qitem[b]{ \Qq{Navigation features:} \Qtab{5cm}{difficult to follow ~~ \Qrating{5} ~~ easy to follow}}

\Qitem[c]{ \Qq{Gene set selection:} \Qtab{5cm}{difficult to follow ~~ \Qrating{5} ~~ easy to follow}}

\Qitem[d]{ \Qq{Power analysis:} \Qtab{5cm}{difficult to follow ~~ \Qrating{5} ~~ easy to follow}}

\minisec{Please evaluate the following video tutorials: \texttt{http://aegis.stanford.edu/tutorial/}.}
\vskip.5em

\Qitem[a]{ \Qq{Video tutorial 1:} \Qtab{5cm}{difficult to follow ~~ \Qrating{5} ~~ easy to follow}}

\Qitem[b]{ \Qq{Video tutorial 2:} \Qtab{5cm}{difficult to follow ~~ \Qrating{5} ~~ easy to follow}}

\Qitem[c]{ \Qq{Video tutorial 3:} \Qtab{5cm}{difficult to follow ~~ \Qrating{5} ~~
easy to follow}}

\Qitem[d]{ \Qq{Video tutorial 4:} \Qtab{5cm}{difficult to follow ~~ \Qrating{5} ~~
easy to follow}}

% \vspace{0.02in}
\Qitem{ \Qq{About how long did it take you to read the documentation and tutorials?}
\Qline{2.5cm} .}

\minisec{Please describe your experience with the search strategies below.}
\vskip.5em

\Qitem[a]{ {\Qq{Could you find all concepts of interests through the auto-complete search?} \Qtab{5.5cm}{\QO{} Yes.
\hskip0.3cm \QO{} No.}}
If not, can you describe what your attempts were?  \Qline{7.8cm} .
}

\Qitem[b]{ {\Qq{What anchors did you use to specify the context graph?} \Qtab{5cm}{~\QO{} waypoint anchors
\hskip0.2cm \QO{} root anchors}}}

\Qitem[c]{ {\Qq{What anchors did you use to specify the focus graph?} \Qtab{5cm}{~~~ \QO{} waypoint anchors
\hskip0.2cm \QO{} leaf anchors}}}

\Qitem[d]{ {\Qq{How was your search result?}} \Qtab{5.5cm}{not informative ~~ \Qrating{5} ~~
very informative}}

\minisec{Please describe your experience with the visualization interactions below.}
\vskip.5em

\Qitem[a]{ {\Qq{What graph layout did you use most frequently?} \Qtab{5cm}{\QO{} root-bound
\hskip0.4cm \QO{} leaf-bound \hskip0.4cm \QO{} buoyant}}}

\Qitem[b]{ {\Qq{What was the average scale of your visualization?} \Qtab{1cm}{\Qline{1cm} context nodes, and \Qline{1cm} focus nodes.}}}

\Qitem[c]{ {\Qq{Were you able to select the appropriate node highlights for your application?} \Qtab{5.5cm}{~ \QO{} Yes.\hskip0.2cm \QO{} No.}}
If not, can you describe what you would prefer to highlight?  \Qline{6.4cm} .
}
\Qitem[d]{ {\Qq{Was the provided the GO annotation information sufficient for your analysis?} \Qtab{5.5cm}{\QO{} Yes.
\hskip0.1cm \QO{} No.}}
If not, can you describe what additional information you would like to see?  \Qline{4.1cm} .
}
% \vspace{0.03in}

\Qitem{ {\Qq{Did you perform power calculations for your study design?} \Qtab{5.5cm}{\QO{} Yes
\hskip0.3cm \QO{} No}}
If yes, how long did it take you to setup it up and wait for it complete?  \Qline{1.95cm} and \Qline{1.95cm} .
}

\Qitem{ \Qq{In case you have additional feedback or suggestions, please write them below.} \Qlines{2}
}

\end{document}
